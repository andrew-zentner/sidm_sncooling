\documentclass[nofootinbib,prd,superscriptaddress,twocolumn]{revtex4}
\usepackage{amsmath,graphicx,hyperref}
\usepackage{slashed}
\usepackage{feynmp}
\usepackage{epstopdf}
\usepackage{color}
\usepackage[dvipsnames]{xcolor}


%%%%%%%%%%%%%%%%%%%%%%%%%%%%%%%%%%%%%%%%%%%%%%%%%%%%%%%%%%%%%%%%%%%%%%
%%%%%%%%%%%%%%%%%%%%%%%%%%%%% Command %%%%%%%%%%%%%%%%%%%%%%%%%%%%%%%%
%%%%%%%%%%%%%%%%%%%%%%%%%%%%%%%%%%%%%%%%%%%%%%%%%%%%%%%%%%%%%%%%%%%%%%

\newcommand{\fms}[1]{{#1}\!\!\!/}
\newcommand{\fmsl}[1]{{#1}\!\!\!\!/}

\newcommand{\mc}{\mathcal}
\newcommand{\mr}{\mathrm}
\newcommand{\mP}{\mathcal{P}}
\newcommand{\mO}{\mathcal{O}}

\newcommand{\be}{\begin{equation}}
\newcommand{\ee}{\end{equation}} 
\newcommand{\beq}{\begin{equation}} 
\newcommand{\eeq}{\end{equation}} 
\newcommand{\bea}{\begin{eqnarray}} 
\newcommand{\eea}{\end{eqnarray}} 

\newcommand{\ov}{\overline}
\newcommand{\pp}{\perp}
\newcommand{\dg}{\dagger}
\newcommand{\n}{\overline{n}}
\newcommand{\nn}{\frac{\fms{\overline{n}}}{2}} 
\newcommand{\nnn}{\frac{\fms{n}}{2}} 

\newcommand{\bl}[1]{{\bf{#1}}}
\newcommand{\bla}[1]{|{\bf{#1}}|}
\newcommand{\blp}[1]{{\bf{#1}}_{\perp}}
\newcommand{\blpu}[1]{{\bf{#1}}^{\perp}}
\newcommand{\blpa}[1]{|{\bf{#1}}_{\perp}|}
\newcommand{\bs}[1]{\boldsymbol{#1}}
\newcommand{\bsp}[1]{{\boldsymbol{#1}}_{\perp}}

\newcommand{\ds}{\displaystyle} 
\newcommand{\nnb}{\nonumber} 

\newcommand{\as}{\alpha_s}
\newcommand{\qa}{q_{\mathrm{A}'}}

\newcommand{\eps}{\varepsilon} 
\newcommand{\veps}{\varepsilon} 

\newcommand{\UV}{\veps_{\mr{UV}}}
\newcommand{\IR}{\veps_{\mr{IR}}}

\newcommand{\La}{\Lambda^2_{\rm{alg}}}

\newcommand{\Aprime}{\mathrm{A}'}
\newcommand{\rdp}{r_{\mathrm{dp}}}

\newcommand{\dd}{\mathrm{d}}

%--- journals
\newcommand{\physrep}{Phys. Reports}
\newcommand{\mnras}{Mon. Not. R. Astron. Soc.}
\newcommand{\apjl}{Astrophys. J. Lett.}
\newcommand{\jcap}{J. Cosmol. Astropart. Phys.}

%--- for comments
\newcommand{\arz}[1]{{{\bf{\color{BrickRed}[ARZ: #1]}}}}

%%%%%%%%%%%%%%%%%%%%%%%%%%%%%%%%%%%%%%%%%%%%%%%%%%%%%%%%%%%%%%%%%%%%%%

\bibliographystyle{apsrev}

%\topmargin 0.0 in

\begin{document}

\baselineskip 3.0ex 

\vspace*{18pt}

%%%%%%%%%%%%%%%%%%%%%%%%%%%%%%%%%%%%%%%%%%%%%%%%%%%%%%%%%%%%%%%%%%%%%%
%%%%%%%%%%%%%%%%%%%%%%%%%%%%% Title %%%%%%%%%%%%%%%%%%%%%%%%%%%%%%%%%%
%%%%%%%%%%%%%%%%%%%%%%%%%%%%%%%%%%%%%%%%%%%%%%%%%%%%%%%%%%%%%%%%%%%%%%

\title{Updated Constraints on Self-Interacting Dark Matter from Supernova 1987A}

\def\Pitt{Pittsburgh Particle Physics Astrophysics and Cosmology Center (PITT PACC) \\ Department of Physics and Astronomy, University of Pittsburgh, Pittsburgh, Pennsylvania 15260, USA}

\author{Cameron Mahoney}
\email[E-mail:]{cbm34@pitt.edu}
\affiliation{\Pitt}
\author{Adam K. Leibovich}
\email[E-mail:]{akl2@pitt.edu}
\affiliation{\Pitt}
\author{Andrew R. Zentner}
\email[E-mail:]{zentner@pitt.edu}
\affiliation{\Pitt}  

\begin{abstract} 
\baselineskip 3.0ex   
Abstract?
\end{abstract}

\maketitle 

%%%%%%%%%%%%%%%%%%%%%%%%%%%%%%%%%%%%%%%%%%%%%%%%%%%%%%%%%%%%%%%%%%%%%%

%------------------------- Introduction ---
\section{Introduction}

An overwhelming preponderance of observational evidence indicates that a form of nonrelativistic, nonbaryonic, 
dark matter constitutes the majority of mass in the Universe and drives the formation of cosmic structure. 
The pace of the quest to identify the dark matter is accelerating on many fronts. Weakly-interacting massive 
particles (WIMPs) have received the most attention as dark matter candidates (see Ref.~\cite{jungman_etal96} for a review). 
Dark matter particles that interact with standard model particles only weakly, while interacting among themselves 
much more strongly have been studied as an alternative to WIMP scenarios in numerous contexts 
\cite{carlson_etal92,deLaix_etal95,atrio-barandela_davidson97,spergel_steinhardt00,hogan_dalcanton00,mohapatra_teplitz00,
dave_etal01,hisano_etal04,hisano_etal05,pospelov_etal08,arkani-hamed_etal08a,lattanzi_silk08,ackerman_etal09,feng_etal09,
kong_etal15} \arz{Cameron: Please review the literature and add some more recent references here. 
I know that Annika Peter has a few papers and Jonathan Feng probably has some more. 
Ensuring that you are up to date on these papers is important.} 
and constraints on self-interacting dark matter (SIDM) models have been explored by many authors \cite{yoshida_etal00,gnedin_ostriker01,miralda-escude02,randall_etal08,kamionkowski_profumo08,zentner09,robertson_zentner09,pieri_etal09,spolyar_etal09,finkbeiner_etal09,
slatyer_etal09,bramante_etal14,albuquerque_etal14,kaplinghat_etal14,chen_etal14,feng_etal16,catena_widmark16}. 
In this paper, we revisit and update astrophysical constraints on SIDM models from 
supernova cooling.


SIDM models in which large self-interaction cross sections are mediated by sufficiently light 
bosons ($M \lesssim 100$~GeV) can be constrained astrophysically using supernovae, particularly 
SN1987A. Light gauge bosons will be produced within the hot supernova core, primarily through 
brehmsstrahlung, and radiated. This non-standard energy loss mechanism can result in an energy loss rate 
from the supernova core that is inconsistent with observations of neutrinos from SN1987A, 
analogous to the classic constraint on axions \cite{turner88,raffelt96_book}. 
This effect has already been exploited by Dent et al. \cite{dent_etal12}, Rrapaj and Reddy \cite{rrapaj_reddy16}, and 
Hardy and Lasenby \cite{hardy_lasenby17} \arz{Cameron, add others of which you may be aware.}
to constraint dark electromagnetism models in which the new gauge boson, the so-called dark photon, 
is kinetically mixed with the standard model photon. The SN1987A constraint places a limit on the mixing parameter. 


We initiated our study because of a number of ambiguities appearing in the previous literature on this subject. 
In particular, we could not reproduce the constraints of Ref.~\cite{dent_etal12}. During the course of our 
study, we identified a number of errors in the analysis of Ref.~\cite{dent_etal12}. First, it is straightforward 
to demonstrate that the kinematical relationships given in Appendix A of Ref.~\cite{dent_etal12} are 
incorrect. Second, the squared matrix elements given in Eq.~(A3) and Eq.~(C18) of Ref.~\cite{dent_etal12} must 
be incorrect. We have not been able to identify the errors which led to these incorrect expressions; however, 
there must have been several oversights in the analysis of Ref.~\cite{dent_etal12}. For example, the squared 
matrix elements in Eq.~(A3) and Eq.~(C18) do not obey the correct symmetries under interchange of incoming 
and/or outgoing momenta. Furthermore, Ref.~\cite{dent_etal12} neglects the mass of the gauge boson, which 
is legitimate in the classic case of the $\sim$meV-mass axion, but not in the present context. Finally, Ref.~\cite{dent_etal12} 
employed an inconsistent model for the permitted energy loss rate from the supernova interior. Our work amounts 
primarily to repeating the calculation of Ref.~\cite{dent_etal12} in order to rectify these oversights. Our primary 
calculation treats nucleon scattering via one-pion exchange (OPE). We have repeated this analysis 
using and effective field theory for the nucleon interaction and find very similar results so we do not 
report on those in detail. \arz{I think we may want to add back in the EFT calculation for this paper. 
That would at least give this paper something a bit unique when compared to the other papers on 
the subject.}


As we were completing our manuscript, two related papers were published. Rrapaj and Reddy \cite{rrapaj_reddy16} 
computed bounds on mixing of the dark and standard model photons using a soft radiation approximation (SRA) 
for dark photon Brehmsstrahlung. This has the distinct advantage of enabling nucleon scattering data to be used 
directly in the estimation of the Brehmsstrahlung rate, but is only approximate because at large dark photon masses 
the radiated dark photons carry off considerable momentum and energy. We have been able to reproduce the 
result of Rrapaj and Reddy \cite{rrapaj_reddy16} and find that the SRA plausibly results in only a factor or 
$\sim 3$ underestimate of the upper bound on the dark photon mixing parameter at most. Moreover, 
our OPE results agree well with the SRA calculation of Rrapaj and Reddy \cite{rrapaj_reddy16}. Nonetheless, 
Ref.~\cite{rrapaj_reddy16} quote results very similar to those of Dent et al. \cite{dent_etal12}. We find, 
rather remarkably, that the various errors in the analysis of Ref.~\cite{dent_etal12} conspire to yield a constraint that 
is very similar to the correct answer. 


More recently, Hardy and Lasenby \cite{hardy_lasenby17} studied bounds on these same models (and others) 
including plasma effects. For simplicity, we have not included these plasma effects in our calculations. Hardy 
and Lasenby base their calculation off of the SRA of Ref.~\cite{rrapaj_reddy16}. Consequently, 
for dark photon masses $\gtrsim 10$~MeV they find very similar results to Ref.~\cite{rrapaj_reddy16} as well 
as the constraints we present in this manuscript. For dark photon masses $\lesssim 10$~MeV, Hardy and 
Lasenby demonstrate that constraints on the dark photon mixing parameter are significantly weaker than 
one finds when neglecting plasma effects \cite{hardy_lasenby17}. 


It is important to delineate correctly the range of the viable parameter space for SIDM models. 
The parameter space available to dark electromagnetism models of SIDM has been studied extensively 
not only in the aforementioned papers, but also in the work of Bjorken et al. \cite{bjorken_etal09} and 
the Snowmass white paper by Kaplinghat, Tulin, and Yu \cite{kaplinghat_etal13_whitepaper}. One 
point that is clear from previous work is that there is at most only a slim sliver of parameter space that 
can simultaneously yield the correct relic abundance of SIDM through thermal production, 
have interesting effects on cosmological structure formation, and 
evade all constraints including the constraints from SN1987A. 
In accord with Ref.~\cite{rrapaj_reddy16} and Ref.~\cite{hardy_lasenby17}, we find that 
the constraints quoted in Ref.~\cite{dent_etal12} are too restrictive, but only by a factor 
of $\sim 4$ due to a conspiratorial cancellation of errors in Ref.~\cite{dent_etal12}. 
Hardy and Lasenby go on to demonstrate that these constraints are significantly too 
restrictive for dark photon masses $\lesssim 10$~MeV due to plasma effects.


The remainder of this paper is organized as follows. In Section~\ref{section:model}, we discuss 
dark photon models. We describe our calculation of SN1987A constraints on SIDM in 
Section~\ref{section:computational} and present our primary results in Section~\ref{section:results}. 
We stress only those points that are key to understanding the relationship between our work 
and the work of both Dent et al. in Ref.~\cite{dent_etal12} and Rrapaj and Reddy in Ref.~\cite{rrapaj_reddy16}. 
We summarize our work and draw conclusions in Section~\ref{section:conclusions}. 


%---- Model Details
\section{Dark Photon Model of SIDM}
\label{section:model}

\arz{Need to add citations to many of the Feng et al. papers here.}
We consider constraints on SIDM specifically within the context of dark electromagnetism models. 
Dark electromagnetism models are models in which a hidden, dark, sector contains a broken U(1)$'$ 
symmetry and the U(1)$'$ gauge boson is kinetically mixed with the standard model photon. For the 
purposes of this study, this is important because it demands that the Lagrangian contains terms such as 
%
\begin{equation}
\mathcal{L}_{\mathrm{int}} = g_{\chi} \bar{\chi} \, \widetilde{\slashed{\Aprime}} \chi + q \bar{f}\, \widetilde{\slashed{A}} f, 
\end{equation}
%
where $\chi$ is the dark matter, $g_{\chi}$ is the dark coupling, $\widetilde{\Aprime}$ is the dark gauge boson, 
$f$ is a standard model fermion of charge $q$, and $\widetilde{A}$ is the standard model gauge boson. 
The kinetic mixing, through a term 
$\frac{1}{2}\frac{\varepsilon}{\sqrt{1+\varepsilon^2}}\, \widetilde{F}_{\mu \nu}\widetilde{F}'^{\mu \nu}$ 
in the Lagrangian causes the $\widetilde{A}$ to be an admixture of the massless photon $A$, 
and the dark photon $\Aprime$, 
of mass $m_{\mathrm{\Aprime}} = m_{\mathrm{\widetilde{A}'}} \sqrt{1 + \varepsilon^2} \simeq m_\mathrm{{\widetilde{A}'}}$ because 
the viable parameter range has $\varepsilon \ll 1$. The dark matter particles are thereby coupled 
to the standard model fermions with a coupling constant $\varepsilon q$, 
where $\varepsilon$ is the kinetic mixing parameter. The first term in this 
interaction Lagrangian gives rise to the dark matter self-interactions. 

Dark gauge bosons are produced in astrophysical environments such as supernova 
cores primarily via brehmsstrahlung off of standard model particles. This brehmsstrahlung 
occurs through the $\varepsilon q$ coupling to charged standard model particles, 
in this  particular case the proton and pion. The rate of brehmsstrahlung 
depends upon both $\varepsilon$ and the mass of the $\Aprime$. Consequently, 
supernova cooling can constrain the mixing $\varepsilon$ as a function of 
$m_{\Aprime}$ for such models. Delineating such a constraint is the 
primary aim of this paper.


%----- Methods
\section{Methods}
\label{section:computational}
	
We aim to estimate the rate of energy loss from the core of a supernova from $\Aprime$ brehmsstrahlung 
during nucleon-nucleon interactions. The calculation is analogous to the well-known estimate of axion 
emission from supernova cores described in Ref.~\cite{raffelt96_book} and references therein, but is 
more complicated because the mass of the $\Aprime$, unlike the mass of the axion, is not necessarily 
negligible. This section describes the calculation of the rate of energy loss from a supernova core 
from $\Aprime$ brehmsstrahlung. 

The brehmsstrahlung process is not the only process with which we must be concerned. Clearly, the rate of 
brehmsstrahlung will increase with $\varepsilon$; however, $\varepsilon$ can become sufficiently large 
that the radiated gauge bosons do not escape the supernova. This happens if the $\Aprime$ particles 
either decay to or interact with standard model particles prior to exiting the supernova core. 
In either case, the energy is not lost and the $\Aprime$ does not provide a cooling channel for the supernova. 
Consequently, for a given $m_{\Aprime}$, there is a maximum $\varepsilon$ that can be constrained in this 
manner. We estimate $\Aprime$ decay and scattering probabilities, and the upper limits on the $\varepsilon$ 
constraints in this section as well. However, we note that terrestrial experiments generally rule out 
mixing parameters higher than the upper limits of the SN1987A forbidden region, so a precise estimate 
of these upper limits is not necessary.

%-- Brehmsstrahlung--	
\subsection{Brehmsstrahlung of $\Aprime$ Bosons}


There are two processes to consider in order to estimate the rate of energy loss via $\Aprime$ brehmsstrahlung. The first is  
proton-proton (pp) scattering with the bremsstrahlung of the dark photon off the proton; $p+p \rightarrow p+p+A$. The 
second is proton-neutron (pn) scattering with bremsstrahlung off of either the proton or the charged pion; 
$p+n \rightarrow p+n+A$. We estimate the rates for these processes using the one-pion exchange (OPE) 
approximation for nucleon interactions. In the pp case, there are eight tree-level diagrams, 
with the emission of the $\Aprime$ from each of the external legs. One of these diagrams is shown 
in Fig.~\ref{fig:ppdiagram}; the remaining seven diagrams come from placing the radiated $\Aprime$ 
on each of the other three protons and then, for each of these, interchanging the outgoing momenta. 
For the pn case, there are five diagrams, four of which are analogous to the pp diagram shown in 
Fig.~\ref{fig:ppdiagram}. The fifth diagram, shown in Fig.~\ref{fig:npdiagram}, corresponds to 
emission of the $\Aprime$ from the exchanged, charged pion.

%-------------- PP Feynman Diagram
\begin{figure}
\includegraphics[width=8cm]{ppdiagram.pdf}
\caption{One of the eight Feynman diagrams for the pp process. Three of the other diagrams are obtained by 
placing the $\Aprime$ on each of the protons in turn. The remaining four diagrams come from swapping the outgoing 
momenta.}
\label{fig:ppdiagram}
\end{figure}
	
Evaluating these diagrams is tedious, but very straightforward. The calculation differs 
from the well-known axion bremsstrahlung calculation, because the mass of the $\Aprime$ boson is not 
necessarily negligible in the kinematic region of interest for supernova explosions.
The correct kinematical relations are 
%
\bea 
p_1 \cdot p_2 &=& M_N^2 - \frac{l^2}{2} - \frac{k^2}{2} + p_2 \cdot \qa,\\
p_1 \cdot p_3 &=& M_N^2 + k \cdot l - \frac{k^2}{2} + p_3 \cdot \qa,\\  
p_1 \cdot p_4 &=& k \cdot l + M_N^2 - \frac{l^2}{2} + p_4 \cdot \qa, \\
p_2 \cdot p_3 &=& M_N^2 - \frac{l^2}{2}, \\ 
p_2 \cdot p_4 &=& M_N^2 - \frac{k^2}{2},\quad \mathrm{and}\\
p_3 \cdot p_4 &=& k \cdot l + M_N^2 - \frac{l^2 + k^2}{2},
\eea
%
where $p_1$ and $p_2$ are the four-momenta of the incoming nucleons, $p_3$ and $p_4$ are the momenta 
of the outgoing nucleons, $\qa$ is the $\Aprime$ momentum, $k=p_2 - p_3$, $l=p_2 - p_4$, and $M_{\mathrm{N}}$ 
is the nucleon mass. These kinematical relations correct the relations in Ref.~\cite{dent_etal12}.

\arz{Cameron, can you double check this? I think there were some errors in the momenta and in the relative 
signs of the diagrams. I tried to fix them. Please go over this carefully.}
The eight diagrams contribute the following eight terms to the pp amplitude, 
%
\begin{widetext}
\begin{eqnarray}
	M_1 &=& \frac{4 M_N}{ m_\pi} \frac{f_{\mathrm{pp}}^2\, e\, \varepsilon}{k^2-m_\pi^2}  \frac{1}{m_{A^\prime}^2 - 2 \qa \cdot p_1}\ 
	\bar{u}(p_4) \gamma_5 u(p_2)\  \bar{u}(p_3) \gamma_5 (\slashed{p}_1 - \qa +M_N)\slashed \epsilon  u(p_1), \\
	M_2 &=& -\frac{4 M_N}{ m_\pi} \frac{f_{\mathrm{pp}}^2\, e\, \varepsilon}{l^2-m_\pi^2}  \frac{1}{m_{A^\prime}^2 - 2 \qa \cdot p_1}\  
	\bar{u}(p_3) \gamma_5 u(p_2)\  \bar{u}(p_4) \gamma_5 (\slashed{p}_1 - \qa +M_N)\slashed{\epsilon} u(p_1),\\
	M_3 &=& \frac{4 M_N}{ m_\pi} \frac{f_{\mathrm{pp}}^2\, e\, \varepsilon}{l^2-m_\pi^2}  \frac{1}{m_{A^\prime}^2 - 2 \qa \cdot p_2}\ 
	\bar{u}(p_3) \gamma_5 u(p_1)\ \bar{u}(p_4) \gamma_5 (\slashed{p}_2 - \qa +M_N)\slashed \epsilon  u(p_2), \\
	M_4 &=& -\frac{4 M_N}{ m_\pi} \frac{f_{\mathrm{pp}}^2\, e\, \varepsilon}{k^2-m_\pi^2}  \frac{1}{m_{A^\prime}^2 - 2 \qa \cdot p_2}\ 
	\bar{u}(p_4) \gamma_5 u(p_1)\ \bar{u}(p_3) \gamma_5 (\slashed{p}_2 - \qa +M_N)\slashed \epsilon  u(p_2), \\
	M_5 &=& \frac{4 M_N}{ m_\pi} \frac{f_{\mathrm{pp}}^2\, e\, \varepsilon}{k^2-m_\pi^2}  \frac{1}{m_{A^\prime}^2 + 2 \qa \cdot p_3}\
	\bar{u}(p_3) \gamma_5 u(p_1)\ \bar{u}(p_4) \slashed{\epsilon} (\slashed{p}_3 + \qa +M_N)\gamma_5 u(p_2),\\
	M_6 &=& -\frac{4 M_N}{ m_\pi} \frac{f_{\mathrm{pp}}^2\, e\, \varepsilon}{l^2-m_\pi^2}  \frac{1}{m_{A^\prime}^2 + 2 \qa \cdot p_3}\ 
	\bar{u}(p_4) \gamma_5 u(p_1)\ \bar{u}(p_3) \slashed{\epsilon} (\slashed{p}_3 + \qa +M_N)\gamma_5 u(p_2),\\
	M_7 &=& \frac{4 M_N}{ m_\pi} \frac{f_{\mathrm{pp}}^2\, e\, \varepsilon}{k^2-m_\pi^2}  \frac{1}{m_{A^\prime}^2 + 2 \qa \cdot p_4}\ 
	\bar{u}(p_4) \gamma_5 u(p_2)\ \bar{u}(p_3) \slashed{\epsilon} (\slashed{p}_4 + \qa +M_N)\gamma_5 u(p_1),\\
	M_8 &=& -\frac{4 M_N}{ m_\pi} \frac{f_{\mathrm{pp}}^2\, e\, \varepsilon}{l^2-m_\pi^2}  \frac{1}{m_{A^\prime}^2 + 2 \qa \cdot p_4}\ 
	\bar{u}(p_3) \gamma_5 u(p_2)\ \bar{u}(p_4) \slashed{\epsilon} (\slashed{p}_4 + \qa +M_N)\gamma_5 u(p_1),
\end{eqnarray}
\end{widetext}
%
where the dark photon polarization is given by $\epsilon^{\nu}$. The squared amplitude is given by 
%
\begin{equation}
\vert \mathcal{M}_{\mathrm{pp}} \vert^2 = \sum_{s_1,s_2}\, \left\vert \sum_{i=1}^{8}\, M_i \right\vert^2, 
\end{equation}
%
where the first summation is over the incoming proton polarizations. These expressions are identical to those 
given in Ref.~\cite{dent_etal12}; however, they do not simplify significantly if the correct kinematics are used. 
Unfortunately, with the kinematical relations above, the squared amplitude does not yield a tidy expression 
for the spin-averaged squared matrix element. Our result 
contains over 200 terms, so we do not reproduce it here for reasons of convenience. However, we note that 
our result for $\vert \mathcal{M}_{\mathrm{pp}} \vert^2$ is symmetric under exchange of $k$ and $l$ as required. 

The pn process contain four diagrams analogous to the the diagrams for the pp process (there are only four, of course, 
because the neutrons cannot radiate the $\Aprime$). The new diagram that is relevant in the pn process is shown in 
Fig.~\ref{fig:npdiagram} and yields a contribution of 
%
\begin{widetext}
\begin{equation}
M'_5 = \frac{4 M_N}{ m_\pi} \frac{f_{pn}^2 e\, \varepsilon}{l^2-m_\pi^2}  \frac{1}{(l-\qa)^2 - m_\pi^2} 
\bar{u}(p_3) \gamma_5 u(p_1) \bar{u}(p_4) \gamma_5 u(p_2) (\qa - 2l)\cdot \varepsilon.
\end{equation}
\end{widetext}
%
The pn processes likewise yield a squared amplitude, 
$\vert \mathcal{M}_{\mathrm{pn}} \vert^2$ that is unwieldy, so we do not give the 
the full expression here. 


%-------------- NP Feynman Diagram
\begin{figure}
\includegraphics[width=8cm]{npdiagram.pdf}
\caption{One of the five Feynman diagrams for the np process. This particular diagram shows 
internal brehmsstrahlung off of the charged pion. The remaining four diagrams are analogous to the 
pp diagram shown in Fig.~\ref{fig:ppdiagram}. In the case of the np processes, there are only four 
diagrams for brehmsstrahlung off of the external legs because two of the legs correspond to the 
uncharged neutron.}
\label{fig:npdiagram}
\end{figure}


%------------------------------------- STREAMING CONSTRAINT
\subsection{The Streaming Limit of the Energy Loss Rate}

The first and simplest bound that may be obtained arises from assuming that all $\Aprime$ particles produced in the supernova core 
leave the supernova, carrying their energies with them. The constraint can be derived simply by 
requiring that the energy loss through this cooling channel be less than the cooling from neutrino emission; 
any greater, and it would have an observable effect on supernova cooling. This calculation yields values of 
$\varepsilon$ above which cooling through $\Aprime$ production is too rapid to be consistent with SN1987A. 
We will consider modifications to this bound from $\Aprime$ trapping and decay in subsequent subsections.

The quantity of interest is the rate of energy emission through dark gauge bosons. 
From the spin-summed, squared amplitudes described in the previous subsection, 
the energy emission rate is obtained by integrating over the phase space, 
and adding a factor of the energy of the emitted particle. To be specific, the 
energy emission rate per unit volume is 
%
\begin{widetext}
\beq
\label{eq:rate1}
Q_i = (2\pi)^4 \int\, E_{\mathrm{A}'} \, \sum_{s_1,s_2}\, \vert \mathcal{M}_i \vert^2 f(p_1) f(p_2)\delta^{(4)}(p_1+p_2-p_3-p_4-\qa)\, \dd \Pi,
\eeq
\end{widetext}
%
where 
%
\begin{equation}
d\Pi = \frac{\dd^3 \vec{q}_{\mathrm{A}'}}{(2\pi)^3 2E_{\mathrm{A}'}}\, \prod_{i=1}^{4}\, \frac{\dd^3 \vec{p}_i}{(2\pi^3) 2E_1}
\end{equation} 
is the Lorentz-invariant phase space interval, 
$ E_{\mathrm{A}'}$ is the energy of the emitted $\Aprime$ boson, 
$f(p)$ are the phase-space densities of the incoming nucleons, and 
the index $i$ refers to either the pp or pn processes. 
The nucleons in the core are comfortably non-degenerate and non-relativistic, 
so the Pauli blocking factor is omitted from Eq.~(\ref{eq:rate1}) and we take 
all nucleons to have a Maxwell-Boltzmann phase-space distribution distribution, 
\beq
f(p) =  \frac{n_b}{2} \left( \frac{2 \pi}{M_N T} \right)^{3/2} \exp \left(-\frac{p^2}{2 M_N T} \right).
\eeq
%
We choose a baryon number density of $n_{\mathrm{b}} \approx 1.8 \times 10^{38}\, \mathrm{cm}^{-3}$ and a 
core supernova temperature of $T = 30\, \mathrm{MeV}$, both of which are typical choices and thought to 
be representative of supernova cores. \arz{Need to put a citation here for the properties of supernova cores. 
I can probably find something in the literature, but please put something here if you have a citation.}


We performed the phase space integrals using the Monte Carlo routines in the {\tt CUBA} library. 
\arz{There needs to be some reference, citation, or some other information give about this library here.} 
We integrated over the momenta $\vec{p}_1$, $\vec{p}_2$, $\vec{p}_3$, and the direction of the 
three-momentum of the radiated boson $\hat{q}_{\mathrm{A}'}$, after fixing $\vec{p}_4$ 
and the magnitude of $\vec{q}_{\mathrm{A}'}$ using the delta functions. 
\arz{Cameron, can you double check this to be sure that it is correct? I rephrased it a bit, 
but I found the earlier phrasing to be a bit ambiguous and I can't remember which choice 
you made.} We used the {\tt suave} method provided within {\tt CUBA}, which combines 
importance sampling and adaptive subdivision, as this method provided the best compromise 
between accuracy and run-time for this particular application.

To obtain the dark gauge boson luminosity from $Q_{\mathrm{pp}}$ and $Q_{\mathrm{pn}}$, 
we assumed that $\Aprime$ production takes place in a stellar core of radius $\sim 1$~km, so that 
the total luminosity of $\Aprime$ is 
%
\begin{equation}
L_{\mathrm{A}'} = V (Q_{\mathrm{pp}} + Q_{\mathrm{pn}})
\end{equation}
%
where $V$ is the volume of the sphere. Clearly, $L_{\mathrm{A}'}$ is proportional to $\varepsilon^2$. 
Writing $L_{\mathrm{A}'} = \varepsilon^2 I_{\mathrm{A}'}(m_{\mathrm{A}'},T)$, as in Ref.~\cite{dent_etal12}, 
and $L_{\nu}$ as the neutrino luminosity, we can write the constraint as 
%
\begin{equation}
\label{eq:bound1}
\varepsilon \lesssim \sqrt{\frac{L_{\nu}}{I_{\mathrm{A}'}(m_{\mathrm{A}'}, T)}}.
\end{equation}
%
Following SOMEBODY \arz{We need a citation here.}, we derive a constraint taking 
$L_{\nu} \approx 10^{53}\, \mathrm{erg/s} \approx 4 \times 10^{37}\, \mathrm{MeV}^2$. 


The constraint derived in this manner from Eq.~(\ref{eq:bound1}) sets the lower limit on the 
exclusion band shown in Figure~\ref{fig:constraint}. We will discuss Fig.~\ref{fig:constraint} 
in more detail below. If all of the $\Aprime$ produced in the core leave the supernova freely, all 
values of $\varepsilon$ higher than those given by Eq.~(\ref{eq:bound1}) would be ruled out. 
However, as we have already mentioned, $\varepsilon$ can become sufficiently large that only a 
negligible amount of energy actually exits the supernova core. For large values or $\varepsilon$, 
this can occur because of either $\Aprime$ decays or $\Aprime$ scattering. These additional considerations place 
an upper limit on the values of $\varepsilon$ for which this constraint applies, 
and we discuss these effects in the next two subsections.


%---------------------------------- DECAY LIMIT
\subsection{The Decay Limit}

The constraint from the streaming limit above provides an upper bound on the allowed coupling 
from the production on $\Aprime$ bosons. In order to function as an effective cooling channel, the 
radiated dark gauge bosons must escape the supernova in sufficient numbers. 
As the coupling increases, so do processes that prevent this escape, 
and so the excluded region has an upper bound, above which the luminosity again drops below the 
neutrino luminosity. The first such limit may be found by considering decay of the dark bosons 
into Standard Model particles. Standard model particles will scatter and thermalize 
on a timescale much shorter than the timescale for the evolution of the core, so decays 
contribute nothing to the cooling. 

\arz{I removed reference to $N_{\rm eff}$ because for us $N_{\rm eff}=1$ (doesn't it? the 
only kinematically-allowed decay is to electron-positron pairs, correct?). I also rephrased a 
little bit. Please review this.}
The dark boson has a typical lifetime of
\begin{equation}
\tau_{\mathrm{A}'} = \frac{3}{\varepsilon^2 \alpha m_{\mathrm{A}'}},
\end{equation}
where $\alpha$ is the fine structure constant. Therefore, in the relativistic limit, 
the typical travel distance to decay is given by 
%
\beq
\label{eq:decaylength}
l = \gamma \tau_{\mathrm{A}'} = \frac{3 E_{\mathrm{A}'}}{N_{eff} m_{\mathrm{A}'}^2 \varepsilon^2},
\eeq
and so the fraction escaping the supernova before decaying may be estimated as (e.g., Ref.~\cite{bjorken_etal09})
\beq
\label{eq:decayfactor}
\exp \left( - \frac{r_{\mathrm{decay}}}{l} \right) = 
\exp \left( - \frac{r_{\mathrm{decay}} m_{\mathrm{A}'}^2 \varepsilon^2}{3 E_{\mathrm{A}'}} \right),
\eeq
%
an approximation which is valid so long as the size of the supernova within which standard model decay 
products can interact and be thermalized is significantly larger than the region within which $\Aprime$ are produced, 
an assumption which should be satisfied comfortably. 
To account for the suppression of gauge boson luminosity due to decays we simply multiply the phase space 
integrand in Eq.~(\ref{eq:rate1}) by the exponential suppression factor, after which the calculation proceeds 
as in the previous subsection. The limit is derived in the same way, 
with the additional complication that $I_{\mathrm{A}'}$ is now a function of $\varepsilon $, in addition to $m_{\mathrm{A}'}$ and $T$. 
The constraint Eq.~(\ref{eq:bound1}), therefore, becomes a transcendental equation that must be solved numerically. 


\arz{I have a few questions here. First, what did you choose for the value or $r_{\mathrm{decay}}$? This must 
be specified. 
 
Second, this treatment is valid under the assumption of relativistic dark photons. However, it is not clear to me that this limit is 
the appropriate limit. This limit is valid when $m_{\mathrm{A}'} < T = 30 \, \mathrm{MeV}$. However, I think that 
the decays only become more important than trapping for masses $m_{\mathrm{A}'} \gtrsim 30\, \mathrm{MeV}$. 
This suggests that we should use the non-relativistic limit here. Do I misunderstand something?}


The luminosity in $\Aprime$ will be an increasing function of $\varepsilon$ until decays suppress the gauge boson luminosity, 
at which point $L_{\mathrm{A}'}$ becomes a rapidly decreasing function of $\varepsilon$. Therefore, the excluded values of 
$\varepsilon$ at a given mass will generally have a lower bound set by the calculations of the previous subsection, and an 
upper bound set by decays. An approximate treatment of the upper bound due to decays, as we present here, is 
sufficient because over almost the entire mass range of interest, larger values of $\varepsilon$ are independently 
excluded by terrestrial beam dump experiments \cite{bjorken_etal09}. Therefore, it is far more important to 
derive an accurate estimate of the {\em lower} boundary of the exclusion region (as was done in 
the previous subsection) than the upper boundary of the exclusion region.

%---------------------------------- TRAPPING	
\subsection{Trapping limit}

\arz{This section needs some work as it is a bit hard to follow.}


The second effect that produces an upper bound on the excluded region 
comes from considering trapping of dark bosons within the supernova. 
With a large enough coupling, the dark photons will thermalize and then 
will be emitted from an approximately spherical ``dark photosphere" at the radial position 
where the $\Aprime$ mean free path becomes larger than the typical size of the supernova core. 
In this case the luminosity is given simply by Stefan's law, 
\beq
\mathcal{L}_t  = 4\pi \rdp^2 T_{\mathrm{A}'}^4 \sigma,
\eeq
where $\rdp$ is now the radius of the emitting shell and $T_{\mathrm{A}'}$ its temperature. 
We estimate the radius of this dark photosphere as $\rdp=10$ km, 
because the density of the supernova drops drastically around that point. 
\arz{We need a citation for this statement, but I do not know one off of the top of my head.}
The bound on the luminosity can then be recast as a bound on $T_{\mathrm{A}'}$, 
%
\beq
T_{\mathrm{A}'} \lesssim 9.6\rm\ MeV.
\eeq
%
That bound can then be translated into the desired bound on the coupling as a function of mass by adopting a 
simple model for the supernova atmosphere, assuming that the particles are emitted from the dark photosphere 
at a point where the optical depth to scattering is $\tau = 2/3 $, and finding the temperature that corresponds 
to that optical depth. 

\arz{In what follow ALL SUBSCRIPTS SHOULD BE MADE CONSISTENT WITH PRD CONVENTIONS. Please 
make all subscripts in roman type using ``mathrm." Please take $A \rightarrow \Aprime$, and so on.}

This is a somewhat involved calculation. First, one needs a model for 
the density and temperature in the supernova. Following \cite{turner88}, 
we assume a simple power-law model for the supernova core, with 
\bea
\rho &=& \rho_{\mathrm{p}} \left( \frac{R}{r}\right )^n,\\
T &=& T_{\mathrm{R}} \left[ \frac{\rho(r)}{\rho_{\mathrm{p}}} \right]^{1/3},
\eea
with $  \rho_{\mathrm{p}}  = 3\times10^{14}\ {\rm g/cm}^3$, $T_{\mathrm{R}} = 30$~MeV, 
and $n = 5$. \arz{What is R?}
The optical depth is given by 
\beq
\tau = \int_{r_x}^{\infty}\, \kappa \rho\,  \dd r,
\eeq
where $ \kappa$ is the opacity.  To find the opacity, we start from the reduced mean 
Rosseland mean opacity 
\beq
\frac{1}{\kappa \rho} = 
\frac{15}{4 \pi^4 T^5}\, \int_{m_x}^{\infty}\, \frac{E_A^2 e^{E_A/T} \sqrt{E_A^2 - m_A^2}}{(e^{E_A/T}-1)^2}\, l_A\, \dd E_A,
\eeq
where $ l_A $ is the mean free path. 


The inverse mean free path can readily be obtained by modifying $ Q_i $, 
the expression for the energy loss rate, as follows: removing the factor of $ E_A $ 
and the phase space integral over $ \qa $, and adding a factor of $ e^{E_A/T} $ for detailed balance. 
\arz{I'm not sure that I follow this somewhat cavalier statement about detailed balance. Can someone 
please explain this to me when we meet. In my earlier rounds of thinking about this, I took that factor 
at face value as given, assuming I'd eventually be able to figure out where it comes from.} 
This gives the inverse mean free path as a function of mass and coupling. 
Again, we perform the required integration numerically. \arz{Do you use the same software package here? 
If so, you must say so.} 
This then allows the calculation of $ \kappa_x $. 
\arz{By $\kappa_{\mathrm{x}}$ I suspect that you mean either $\kappa_{\mathrm{pp}}$ or 
$\kappa_{\mathrm{pn}}$. However you never say this. You should state what you mean clearly and explicitly.  
In this particular case, it is probably more cumbersome to define $\kappa_{\mathrm{x}}$ than it is worth because you 
will only use it one time.} 
The inverse opacities for the $pn$ and $pp$ processes add, 
giving the total opacity $ \kappa^{-1} = \kappa_{pp}^{-1} + \kappa_{pn}^{-1} $

	
Having obtained an expression for $ \kappa $, we can now find the optical depth as follows. 
Define a typical optical depth as $\tau_R = \kappa_R \rho_R R $. We then have 
\beq 
\kappa \rho R = \tau_R \left( \frac{\rho}{\rho_R} \right)^2 \left( \frac{T_R}{T} \right)^{3/2}.
\eeq
%
This can be combined with the previous expressions for the density and temperature as a function of position 
and plugged in to the integral expression for the optical depth to obtain 
\bea 
\tau_x &=& \int_{r_x}^{\infty}\, \tau_R \left( \frac{R}{r} \right)^{3n/2} \dd r \\
 &=& \frac{\tau_r}{\frac{3n}{2}-1} \left( \frac{T_A}{T_R} \right)^{(9/2-3/n)}. 
 \eea
 The bound on the coupling is obtained by requiring  $ \tau_x(\varepsilon,m_A) \le 2/3 $
 \arz{Is it possible to include a few further details here? 
 
 Also, it seems as though you can check the consistency of your assumptions can 
 be checked within this calculation. For example, you can actually use this calculation 
 to estimate the radius at which $\tau=2/3$ and see if it is actually close to 
 $10$~km. Have you done this?
 }
 

As we stated in the previous subsection, the approximate treatment of the upper limit of the exclusion region 
is justified by the fact that values of $\varepsilon$ close to the upper limit of the exclusion region are 
ruled out by independent, terrestrial experiments.


%-------------------- RESULTS
\section{Results}
\label{section:results}



\begin{figure*}[th]
\includegraphics[width=14cm]{constraint_edited.png}
\caption{
Constraints on dark photon models. 
}
\label{fig:constraints}
\end{figure*}

The considerations of the previous section lead to constraints on the 
viable parameters for dark electromagnetism models of SIDM. 



%--------------------------- CONCLUSIONS
\section{Discussion and Conclusions}
\label{section:conclusions}


	The revised approach produce constraints that are significantly weaker than the previous work, and that largely reproduce constraints already obtained from beam dump experiments. 




\acknowledgments

AKL was supported in part by NSF grant PHY-1519175.


\bibliography{draft}

\end{document}

%%%%%%%%%%%%%%%%%%%%%%%%%%%%%%%%%%%%%%%%%%%%%%%%%%%%%%%%%%%%%%%%%%%%%%
%%%%%%%%%%%%%%%%%%%%%%%%%%%%% Bibliography %%%%%%%%%%%%%%%%%%%%%%%%%%%
%%%%%%%%%%%%%%%%%%%%%%%%%%%%%%%%%%%%%%%%%%%%%%%%%%%%%%%%%%%%%%%%%%%%%%


%%%%%%%%%%%%%%%%%%%%%%%%%%%%%%%%%%%%%%%%%%%%%%%%%%%%%%%%%%%%%%%%%%%%%%

\end{document}