\documentclass[]{article}
\usepackage[latin1]{inputenc}
\usepackage{amsmath}
\usepackage{amsfonts}
\usepackage{amssymb}

\begin{document}


\section{EFT Discussion}

	The previous work modelled the nucleon-neucleon interactions involved using the one pion exchange potential (OPE). However, for low energy NN scattering, the OPE approximation proves inconsistent with the nuclear forces involved, and does not represent the fine tuning of the problem. In this case it becomes appropriate to search for an effective field theory (EFT) of the nuclear interaction instead. To this end there has been considerable work done on chiral perturbation theory, producing a number of EFTs representing the low energy interactions. For our purposes we employed perhaps the simplest such scheme, the so-called pionless EFT, in which even the pionic degree of freedom is integrated out, and we are left with a simple contact interaction. The breakdown scale for this theory is set by the pion mass, $ m_\pi = 139 MeV $, and for momenta above this scale in general the pions should be included explicitly in the theory. In our case though, the relevant scales - the supernova temperature and the light dark boson mass - are in general much lower than $ m_\pi $, so we can safely ignore them. There was a slight concern that at the higher end of the mass region we consider, the pionless EFT should no longer apply, but experimentally the EFT we use remains fairly good up to a couple of multiples of $ m_\pi $, by which point Boltzmann suppression factor is so large that EFT deviations become largely insignificant.
	
	To lowest order, the pionless EFT effective interaction Lagrangian is \[ {\cal L} =
	C_{0}^{\left( ^{3}S_{1}\right) }(N^{T}P_{i}N)^{\dagger }(N^{T}P_{i}N) + C_{0}^{\left( ^{1}S_{0}\right) } (N^{T}\overline{P}_{i}N)^{\dagger }((N^{T}\overline{P}_{i}N) \] where $ C_{0}^{\left( ^{3}S_{1}\right) }, C_{0}^{\left( ^{1}S_{0}\right) } $ are the effective couplings for the $ \left ^{3}S_{1}\right $ and $ {\left ^{1}S_{0}\right $ channels respectively. $ P_i, \overline{P}_i $ are the projection operators onto each channel, explicitly \[ P_{i}= \frac{1}{\sqrt{8}} \sigma_2 \tau_2 \tau_i, \overline{P}_i = \frac{1}{\sqrt{8}} \sigma_2 \sigma_i \tau_2 \], where $ \sigma $ ($ \tau $) acts on spin (isospin) space. The effective couplings are determined by matching to the effective range expansion, and are renormalization-dependent.
		% Can go into much more detail about power counting and matching conditions, but didn't think it was appropriate
	The diagrams to consider therefore has only external fermion lines, with the gaugue boson coming off one of them. This is considerably simpler than the full OPE calculation. 
	


\end{document}
